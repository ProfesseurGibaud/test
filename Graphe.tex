%----------------------------------------------------------------------------------------
%	PACKAGES AND OTHER DOCUMENT CONFIGURATIONS
%----------------------------------------------------------------------------------------

\documentclass[12pt,fleqn]{article} % Default font size, left-justified equations, chapters start on any page

%----------------------------------------------------------------------------------------
\input{structure}
\input{structureSG} % Insert the commands.tex file which contains the majority of the structure behind the template



%----------------------------------------------------------------------------------------

\begin{document}


\title{TP Guidé : Les Graphes}
\maketitle

\section{Définitions et Premières Propriétés}

\begin{definition}
	Un graphe $G$ est la donnée d'un ensemble de nœuds noté $V$ et d'un ensemble d'arêtes reliant ces nœuds noté $E$. On écrit $G = (V,E)$
\end{definition}

\begin{remark}
	Les notations $V$ et $E$ viennent de l'anglais. $V$ signifie Vertex qui veut dire Sommet. $E$ signifie Edge qui veut dire arête.
\end{remark}

\begin{example}
	Le graphe $G = (V,E)$ avec $V = \lbrace A,B,C,D \rbrace$ et $E = \lbrace [A,B],[A,C],[A,D],[D,C] \rbrace$ ressemble à la figure suivante.
	\begin{center}
		\begin{tikzpicture}[bend angle=20]
			\tikzstyle{db}=[circle, draw , inner sep=0pt, minimum width=6pt]
			\tikzstyle{fleche}=[draw, thick, -,=latex]
			\node[db](A) at (0,0){} node[below]{$A$};
			\node[db](B) at (1.5,1.2){};
			\node[db](C) at (0,1.5){}; \node [db](D) at (-1,1){};
			\draw [-,=latex, thick](A) to [bend right] (B) ;
			\draw[-,=latex, thick](A) to [bend left] (D) ;
			\draw[-,=latex, thick] (C)--(A);
			\draw[-,=latex, thick] (D) to [bend left=60](C);
			\draw(1.5,1.2)node[right]{$B$};
			\draw(0,1.5)node[above]{$C$} ;
			\draw(-1,1)node[left]{$D$}
		\end{tikzpicture}
	\end{center}
\end{example}

\begin{exercise}
	\begin{enumerate}
		\item Faire une classe Noeud avec pour attribut Nom.
		\item Faire une classe Arête avec pour attributs Noeud1 et Noeud2
		\item Faire une classe Graphe avec pour attributs : une liste comportant des objets de type Noeud et une liste comportant des objets de type arête.
		\item Construire 2 exemples : $G = ([1,2,3],([1,2],[1,3]))$ et \\${GG = ((["A","B","C"]),(["A","C"],["B","C"]))}$
	\end{enumerate}
\end{exercise}


Il existe un outil mathématique utile pour manipuler ces graphes (et réaliser des modèles). Cet outil est la matrice d'adjacence. Il permet en un seul tableau de nombre de se faire une représentation du graphe (une matrice est un tableau de nombre).
\begin{definition}
	Soit $G = (V,E)$ un graphe, on peut construire une matrice telle que les entrées de la matrice sont les noeuds et à la case $(i,j)$ on met \begin{itemize}
		\item 1 si il y a une arête entre $i$ et $j$
		\item 0 sinon
	\end{itemize}
\end{definition}

\end{document}